% Template for PLoS
% Version 1.0 January 2009
%
% To compile to pdf, run:
% latex plos.template
% bibtex plos.template
% latex plos.template
% latex plos.template
% dvipdf plos.template

\documentclass[10pt,draft]{article}

% amsmath package, useful for mathematical formulas
\usepackage{amsmath}
% amssymb package, useful for mathematical symbols
\usepackage{amssymb}

% graphicx package, useful for including eps and pdf graphics
% include graphics with the command \includegraphics
\usepackage{graphicx}

% cite package, to clean up citations in the main text. Do not remove.
\usepackage{cite}

\usepackage{color} 

% Use doublespacing - comment out for single spacing
%\usepackage{setspace} 
%\doublespacing


% Text layout
\topmargin 0.0cm
\oddsidemargin 0.5cm
\evensidemargin 0.5cm
\textwidth 16cm 
\textheight 21cm

% Bold the 'Figure #' in the caption and separate it with a period
% Captions will be left justified
\usepackage[labelfont=bf,labelsep=period,justification=raggedright]{caption}

% Use the PLoS provided bibtex style
\bibliographystyle{plos2009}

% Remove brackets from numbering in List of References
\makeatletter
\renewcommand{\@biblabel}[1]{\quad#1.}
\makeatother


% Leave date blank
\date{}

\pagestyle{myheadings}
%% ** EDIT HERE **


%% ** EDIT HERE **
%% PLEASE INCLUDE ALL MACROS BELOW

%% END MACROS SECTION

\begin{document}

% Title must be 150 characters or less
\begin{flushleft}
{\Large
\textbf{Digital Normalization of Short Shotgun Sequences Facilitates
{\em de novo} Sequence Assembly}
}
% Insert Author names, affiliations and corresponding author email.
\\
C. Titus Brown$^{1,\ast}$, 
Adina Howe$^{2}$,
Qingpeng Zhang$^{3}$,
Alexis B. Pyrkosz$^{4}$,
Timothy H. Brom$^{3}$
\\
\bf{1} Departments of Computer Science and Engineering/Microbiology and Molecular Genetics, Michigan State University, East Lansing, MI, USA
\\
\bf{2} Departments of Microbiology and Molecular Genetics/Crop and Soil Sciences, Michigan State University, East Lansing, MI, USA
\\
\bf{3} Department of Computer Science and Engineering, Michigan State University, East Lansing, MI, USA
\\
{\bf{4} USDA Avian Disease and Oncology Laboratory, East Lansing, MI, USA}
\\
$\ast$ E-mail: Corresponding ctb@msu.edu
\end{flushleft}

% Please keep the abstract between 250 and 300 words
\section*{Abstract}

{\bf Background:} Deep shotgun sequencing and analysis of genomes,
transcriptomes, amplified single-cell genomes, and metagenomes enable
the sensitive investigation of a wide range of biological
phenomena. However, it is difficult to deal with the volume of data
emerging from deep short-read sequencers, in part because of random
and systematic sampling variation as well as many sequencing errors.
These challenges have led to the development of entire new classes of
short-read mapping tools such as Bowtie and BWA, as well as new de
novo assemblers such as ABySS, Velvet, SOAPdenovo, ALL-PATHS, and SGA.
Even newer assembly strategies for dealing with transcriptomes,
single-cell genomes, and metagenomes have also emerged.  Despite these
advances, algorithms and compute capacity continue to be challenged by
the continuing improvements in sequencing technology.
\\
\\
{\bf Methodology and Principal Findings:} We describe an approach we call
digital normalization, a single-pass computational algorithm that
reduces sampling variation and eliminates the majority of sequencing
errors from deep sequencing data. Digital normalization substantially
reduces the size of the data set and decreases the memory and time
requirements for {\em de novo} sequence assembly without significantly
impacting content of the generated contigs.  Moreover, for at least
one single-cell data set, the post-normalization assembly exhibits a significant improvement in contiguity over previous best methods.
\\
\\
{\bf Conclusions and Significance:} The digital normalization approach
systematically reduces data set size and removes errors prior to {\em
de novo} sequence assembly.  We demonstrate its
applicability to the assembly of microbial genomes, single-cell
genomic data, and transcriptomic data.

% Please keep the Author Summary between 150 and 200 words
% Use first person. PLoS ONE authors please skip this step. 
% Author Summary not valid for PLoS ONE submissions.   
\section*{Author Summary}

\section*{Introduction}

The ongoing revolution in DNA sequencing technologies has led to a new
problem: how do we analyze the resulting very large sequence data sets
quickly and efficiently? These data sets contain millions to billions
of short reads, and have high error rates and substantial sampling
biases. In particular, the widespread use of deep shotgun sequencing
on previously unsequenced genomes, transcriptomes, and metagenomes has
led to vast quantities of high-coverage sequencing data that requires
de novo assembly because there is no existing reference genome.

There are two basic challenges to dealing with short-read sequences
from shotgun sequencing. The first is that deep sequencing is needed
for thorough sampling. This is because shotgun sequencing samples
randomly, and always from a distribution biased by sample content,
sample preparation and sequencing technology.  For this reason, a
human genome may require 100x coverage or more for near-complete
sampling, leading to 300 gigabase+ data sets.  Since sampling depth
follows the prevalence of the lowest abundance molecule, transcriptomes and
metagenomes also require similarly deep sequencing for complete
sampling of rare molecules. This leads to extremely large amounts of
data, much of it redundant.

The second problem with short-read shotgun sequencing is the high
error rate.  Illumina GAII has up to a 1\% error rate in its sequences,
leading to one error in every 100 bp of data.  Because this error rate
is constant, the number of errors in the data set grow linearly with
the amount of data generated.  These errors dominate novelty in
sequencing data sets.

These two problems have combined to result in a significant
computational challenge: it is now straightforward to generate data
sets that cannot easily be analyzed computationally.  While approaches
for scaling bioinformatic analysis are emerging, sequencing capacity
is continuing to grow faster than increases in computational capacity,
leading to an arms race that cannot be won by simply improving compute
hardware.  What algorithmic solutions have been applied?

Analysis of the large amounts of data resulting from deep sequencing
has been tackled in a number of ways.  A whole new class of alignment
tools, mostly relying on the Burroughs-Wheeler transform, has been
created specifically to do ultra-fast short-read alignment to
reference sequence.  In cases where the reference sequence must be
assembled de novo from the sequence data, a number of new assemblers
have been created, including ABySS, Velvet, SOAPdenovo, ALLPATHS, and
SGA.  These assemblers rely on theoretical advances such as the
application of de Bruijn graphs and compressed @LM data structures to
store large amounts of data in a computationally convenient manner.
As short-read sequencing has been applied to single cell genomes,
transcriptomes, and metagenomes, yet another generation of assemblers
has emerged to handle reads from very abundance-biased populations of
molecules; these tools, including Trinity, Oases, MetaVelvet,
Meta-IDBA, and Velvet-SV, adopt local models of sequence coverage to
help build assemblies.  While these tools all perform well on large
data sets, their performance is extremely sensitive to the number of
errors present in the underlying data.

Several approaches have been applied to detecting and removing or
correcting errors in large data sets.  Typically, these approaches
involve looking for fixed-length words with low counts and treating
them as likely errors; these errors can then be eliminated or
``corrected'' by changing them to similar words that are in high
abundance.  One challenge for all of these techniques is that they
require at least two complete iterations across the data set: one
iteration to count all fixed-length words, and a second to perform the
trimming or correction.  For extremely large data sets, this
requirement presents a significant problem: the ``noise'' from
sequencing errors grows linearly with the volume of data, so as data
set size doubles, the computation and memory needed to track
fixed-length words also doubles.

Here, we describe a single-pass algorithm for detection and
elimination of redundant reads in data sets.  This algorithm, which we
call digital normalization, was initially designed to deal with reads
from very abundance-biased samples such as single-cell amplified
genomic DNA, transcriptomes, and mRNAseq.  By eliminating redundant
reads, errors in those reads do not accumulate, and we see a
correspondingly substantial reduction in both data size and total
number of errors.  We present a constant-memory implementation of this
approach that operates in linear time, and demonstrate its
effectiveness for reducing compute requirements for de novo assembly
on a variety of simulated and real data sets.

% Results and Discussion can be combined.
\section*{Results}

\subsection*{Median k-mer abundance of a read is a good estimator of coverage}

Suppose that we have a file of short reads that are randomly sampled
without error from a non-repetitive source genome.  Can we estimate
the coverage of a source genomic region without using an assembled
reference?

One approach to doing so is to select a read from the region of
interest, and then apply the following procedure. First, choose a k
such that all k-mers in the source genome are unique.  Next, for each
k-mer in the selected read, calculate the number of times that that
k-mer appears in {\em any} read in the data set.  The abundance of
these k-mers will represent the true depth of sampling of the source
genome for each k-mer.  We can then estimate the sampling rate or
coverage of the region to which the read belongs by applying a summary
statistic such as the minimum, maximum, median or average of the
read's k-mer abundance distribution.

We can extend this estimation technique to reads sampled {\em with}
error by observing that each single-base miscall or indel changes no
more than k k-mers in the read.  If the errors are random and k is
large, most of these k-mers will be novel in the read data set.  So,
if we order the k-mers in a read by their abundance across the entire
data set, only the abundance of the first k k-mers will be affected by
a single error, and at most the first (2k) k-mers will be affected by
a pair of errors.  The median of the k-mer abundance distribution will
have rank $(l - k + 1) / 2$, which is greater than $k$ for $l > 3k-1$,
and greater than $2k$ for $l > 5k-1$.  Thus for read lengths of 76
bases or longer, and a k-mer size of 20, the median k-mer abundance
will be robust to a single error; for a read length of 100, and a
k-mer size of 20, the median k-mer abundance will be robust to two
errors.

We can therefore use the median k-mer abundance of a read, measured
across the entire read data set, as an estimator of sampling depth for
the region represented by the read -- as long as the genome is
non-repetitive.  Do we actually observe this?  In Figure \ref{fig:random},
we show
that for simulated 100-base reads from an artificial (non-repetitive
genome) of 1 mb, sampled with a 1\% per-base error rate, the median
20-mer abundance matches both the true coverage and the estimated
coverage obtained from mapping the reads back to the genome with
bowtie.  Can we apply this to real genomes?

When we apply the median k-mer abundance measure to real reads sampled
from a real E. coli genome, we see the same correlation (r=XXX)
between median k-mer abundance and estimated coveraged based on mapping.
(Note that here we are using multi-mapping, in which each read is mapped
to all matching locations.)

This same correlation also holds for reads sampled from a non-uniform
distribution, e.g. for mRNAseq expression assays or metagenomics.
Figure \ref{fig:transcripts} shows that median k-mer abundance and number of mapped reads
correlate for a simulated transcriptome with expression across 3
orders of magnitude, while Figure 2b shows the same strong correlation
for a mouse mRNAseq data set.

For what data sets would we expect this correlation between median
k-mer abundance and "true" read coverage to break down?

One biological situation in which we would expect to see divergence
between median k-mer abundance and actual read coverage is for a
highly polymorphic diploid or polyploid sequence, or, more generally,
from a population of molecules with a high rate of internal sequence
variation.  This is because k-mers are notoriously sensitive to high
rates of true variation: for k=20, a diploid sequence with a 5\%
polymorphism rate would generate more than half again as many unique
20-mers as were in a haploid sequence.

Another biological situation is that of repetitive sequence or splice
variants, in which many source molecules will contain identical
k-mers.  When an accurate reference sequence set is available and the
data set is uniformly sampled (e.g. for genome sequencing) it is
straightforward to correct for k-mers that occur multiple times in the
source molecules, by simply averaging each k-mer's counts in the reads
over the actual number of times the k-mer occurs in the reference data
set.  However, this approach has two limitations: first, it cannot be
used in the absence of a high-quality reference sequence; and second,
it cannot be applied when the source data set has a biased abundance
distribution, for example in transcriptomes.

%Discuss fail: repeats, isoforms. Can we construct a fail data set? Has
%to be both repeats and biased abundance.  Correct for # of k-mers.

\subsection*{The median k-mer abundance can be used to remove redundant reads.}

We can now use the median k-mer abundance to progressively remove
reads that do not add significantly to coverage.  Suppose we treat the
read data set as an incoming stream of data consisting of reads which
we can individually accept or reject.  For each additional read, we
use the median k-mer abundance to estimate the coverage for the reads
accepted so far; if this coverage is below a specified threshold, we
accept the read, and otherwise we reject it.  Rejected reads are
discarded and do not contribute to the k-mer counts, while accepted
reads are both saved and incorporated into the k-mer count database.

Algorithmically, we execute the following procedure, which we term
digital normalization:

\begin{verbatim}
   for read in dataset:
      if estimated_coverage(read) < C:
         accept(read)
      else:
         discard(read)
\end{verbatim}

where only accepted reads contribute to the estimated coverage.  For any
data set, this has the effect of discarding the majority of reads that
contribute to coverage $>$ C, although some later reads will be accepted
that push the estimated coverage of earlier reads above C.  Reads from
undersampled regions, or those containing many distinct errors, will always
be accepted and retained.

In Figure 3@@, we display the estimated coverage of a simulated genome
and an E. coli genome for both raw and normalized data, using both
median k-mer abundance and read multimapping.  From this, we see that
digitally normalizing both simulated and real high-coverage data has
the expected effect of normalizing the estimated coverage of the reads
to C, not only for median k-mer abundance but for read multimapping.
For both simulated and real data, then, digital normalization
downsamples the data set to only a subset of reads -- those reads that
contribute to real coverage.  Digital normalization thus converts {\em
random sampling} to {\em systematic sampling}.  What is the effect on
errors?

\subsection*{Digital normalization removes many sequencing errors}

A significant problem with next-generation sequencing is that there
are relatively high per-base error rates, approaching 1-2\%.  These
errors strongly affect the total number of k-mers; for example, in the
simulated genomic data of 100x, a 1\% error rate leads to more than
3-fold more 20-mers in the reads than are truly present in the genome.
This in turn dramatically increases the memory requirements for
tracking and correcting k-mers (cite bloom filter paper).

When we perform digital normalization on such a data set, we eliminate
the vast majority of these k-mers (Table 1).  Intuitively, this is
because we are accepting or rejecting individual reads; in going from
100x random coverage to 20x systematic coverage, we are discarding
80\% of the reads with 62\% of the errors (first row, Table \ref{tab:normC20}).
For reads taken from a biased abundance distribution, such as with
mRNAseq, we also discard many reads, and hence many errors.
Conveniently, in doing so we retain nearly all real k-mers.

During the initial data set reduction we also count the k-mers in the
accepted reads.  With this information we can execute a second pass
across the accepted reads in which we trim reads at low-abundance
k-mers, thus removing another set of likely errors.  Following this
second pass error reduction, we may also execute a second round of
digital normalization that further eliminates redundant data; this
three-pass protocol eliminates some additional errors, at the cost
of very few real k-mers (Table \ref{tab:normC5}).

\subsection*{Digital normalization can be used prior to de novo assembly}

We applied the three-pass digital normalization and error correction
to a variety of real data sets, including sequences from high-coverage
single colonies, single-cell MDA-amplified genomes, and a mouse
transcriptome.  We then ran the Velvet de novo assembler on the
resulting reads, and compared the sequence content of the resulting
assemblies to the known reference, if available.  In all cases with
a reference, digital normalization discarded well over 90\% of the
reads as redundant, while the resulting assemblies were over 95\%
identical to the known reference.

(further discussion of results)

(memory/time calculation.)

@Do we need to look at mapped reads after assembly to verify uniformity
of coverage? Also, we can look at kmer content of read source (for
simulated), reads, and assemblies.

\section*{Discussion}

\subsection*{Converts random sampling to systematic, reference free.}

\subsection*{Digital normalization is an effective method of data reduction}

Digital normalization is a simple, lightweight computational technique
for removing redundant data from shotgun sequencing data sets.  It is
similar to experimental normalization, which has been applied to both
mRNA and single-cell amplified genomes, but is applied after
sequencing.  While there is one disadvantage -- deep sequencing must
be done to sample rare molecules -- this approach confers several
advantages, including the ability to estimate source molecule abundances by
using the unnormalized reads.

Digital normalization uses a reference-independent estimator of
coverage, so it can be applied de novo to any data set.  The approach
is extremely convenient: the runtime complexity is linear with respect
to the data size, and perhaps more importantly it is {\em
single-pass}: the algorithm does not need to look at any read more
than once.  Moreover, because reads are accumulated sub-linearly with
over any deep data set, likely errors also do not accumulate linearly,
and the overall memory usage is consequently limited.  Combined, these
features let us adopt a streaming approach to the data reduction of
extremely large data sets, with all of the advantages of treating
sequence as a data stream (@@).

\subsection*{Digital normalization dramatically scales de novo assembly}

The results from applying digital normalization to read data sets
prior to de novo assembly are extremely good: digital normalization
can reduce data sets by more than 10 in size, without substantially
affecting the assembly results.  Why is this?

First, by discarding reads that do not contribute significant
additional information, digital normalization removes the majority of
reads without affecting the real information content of the read data
set.  For de Bruijn graph assemblers such as Velvet, this reduces the
storage requirements for loading and tracking reads.

Second, by discarding reads, digital normalization eliminates
associated sequencing errors.  These sequencing errors add
significantly to memory usage for assemblers like Velvet, so
eliminating them reduces the memory required for assembly.

Third, by converting random sampling to systematic sampling and
normalizing the coverage, digital normalization eliminates both
sampling bias and variation in read coverage.  This allows
single-genome assemblers such as Velvet to be used on sequences from
samples with biased source distributions such as mRNAseq and
MDA-amplified genomes, as well as metagenomic samples from
mixed-abundance populations. @@underappreciated.

\subsection*{Future challenges}

The two obvious failure modes for digital normalization -- situations
in which the median k-mer abundance will not match true coverage --
are high polymorphism rates and samples with repetitive sequences,
including splice variants.  These are essentially opposite sides of
same coin: polymorphism will result in coverage estimates that are too
low, while repetitive sequence will result in coverage estimates that
are too high.

In practice, there are a number of practical solutions for sequences
from samples with a high rate of polymorphism, depending on the
analysis goals and the underlying biology.

If the end analysis goal is to provide a single reference sequence via
de novo assembly -- a notoriously challenging assembly problem -- it
should be possible to adapt k-mer based read-correction techniques to
such data, and digital normalization could be used as an initial data
reduction step.

If the end analysis goal is to provide the sequence of all strains or
haplotypes via de novo assembly -- also extremely challenging -- some
effective approaches have been demonstrated in the field of
metagenomics.

Repeats are likely to be a more significant problem, because they are
preferentially eliminated by digital normalization.  This results in a
significant underrepresentation of repeats in the normalized read data
set, and may also confuse assemblers that rely on k-mer abundance to
locate repeats.  In this work, we have avoided analyzing or assembling
highly repetitive genomes.  Generally, effective resolution of repeats
usually requires paired-end sequences from several libraries with
different insert sizes, which we do not have for any of the data sets
analyzed here.  For now, we note that our assembly results compare
well with assembler results on non-normalized data, suggesting that
the effect of digital normalization is minimal on the assembly of
short-insert libraries.

The effect on isoforms is undetermined.  @@  Further work!

\subsection{General thoughts}

Digital normalization is an effective demonstration of the basic
observation that at most 2x coverage is required to assemble a genome
-- it just has to be the {\em right} 2x coverage.  Here we have chosen
to normalize samples to 5x, primarily to ensure > 2x systematic
coverage in the face of sequencing errors.  Assembly of normalized
data results in nearly identical assemblies, but with a significant
savings in the required compute time and memory.

We have shown that the median k-mer abundance is a good estimator of
coverage for some samples, but more generally any reference-free
estimator could be used to the same end.  k-mer-abundance based
estimators are easy to calculate, but sensitive to errors and
variation.  Estimators based on local alignments may be worth
investigating in the future.

As sequencing costs continue to drop, we confront essentially
unlimited amounts of data that must be analyzed computationally.
Digital normalization may provide a fast, simple way to reduce the
total volume of data and make certain types of computation more
tractable.  This is especially important for populations of molecules
with strongly biased abundance distributions, for which the
lowest-abundance molecule determines the sampling depth, such as
mRNAseq and metagenomic samples.  Here, digital normalization will
dramatically reduce the overall data size by filtering out data and
noise from the over-sampled molecules, potentially increasing
sensitivity.

\subsection{Misc foo}

Connections to compressed sensing?

Can quantitate off of discarded data.

Ref to experimental normalization

The original data set can be retained for quantification against the
assembled data set.

repeats, scaffolds

Talk about normalization helping basic assemblers.

missassembly analysis/a5
a5 refs

NOT just error removal: smooths things out.

stress preprocessing; attacks de Bruijn principles.

ht size scales with original genome size

show graph of accumulation of reads and errors for simulated data set

show graph of read/kmer rank abundance

% You may title this section "Methods" or "Models". 
% "Models" is not a valid title for PLoS ONE authors. However, PLoS ONE
% authors may use "Analysis" 
\section*{Materials and Methods}

% Do NOT remove this, even if you are not including acknowledgments
\section*{Acknowledgments}

Chris Hart, James M. Tiedje, USDA, NSF.

%\section*{References}
% The bibtex filename
\bibliography{template}

\section*{Figure Legends}
\begin{figure}[!ht]
\begin{center}
\includegraphics[width=4in]{diginorm-fig1a.pdf}
\end{center}
\caption{
{\bf Mapping and k-mer coverage measures correlate for uniform sampling.}
XXX
}
\label{fig:random}
\end{figure}

\section*{Figure Legends}
\begin{figure}[!ht]
\begin{center}
\includegraphics[width=4in]{diginorm-fig2a.pdf}
\end{center}
\caption{
{\bf Mapping and k-mer coverage measures correlate for biased samples.}
XXX
}
\label{fig:transcripts}
\end{figure}

\section*{Tables}

\begin{table}[!ht]
\caption{
\bf{Digital normalization to C=20 removes erroneous k-mers}}
\begin{tabular}{|l|c|c|c|c|}
Data set & True 20-mers & 20-mers in reads & 20-mers at C=20 & \% reads kept\\
\hline \\
Simulated genome & 399,981 & 8,162,813 & 3,052,007 (-2) & 19\% \\
Simulated mRNAseq & 48,100 & 2,466,638 (-88) & 1,087,916 (-9) & 4.1\% \\
{\em E. coli} genome & XXX & YYY & ZZZ & 11\% \\
Mouse mRNAseq & XXX & YYY & ZZZ & 28\% \\
\end{tabular}
\begin{flushleft} XXX
\end{flushleft}
\label{tab:normC20}
\end{table}

\begin{table}[!ht]
\caption{
\bf{Three-pass digital normalization removes most erroneous k-mers}}
\begin{tabular}{|l|c|c|c|c|}
Data set & True 20-mers & 20-mers in reads & 20-mers remaining & \% reads kept\\
\hline \\
Simulated genome & 399,981 & 8,162,813 & 453,588 (-4) & 5\% \\
Simulated mRNAseq & 48,100 & 2,466,638 (-88) & 182,855 (-351) & 1.2\% \\
{\em E. coli} genome & XXX & YYY & ZZZ & 2.1\% \\
Mouse mRNAseq & XXX & YYY & ZZZ & AA\% \\
\end{tabular}
\begin{flushleft} XXX
\end{flushleft}
\label{tab:normC5}
\end{table}

\end{document}

