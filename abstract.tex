% Template for PLoS
% Version 1.0 January 2009
%
% To compile to pdf, run:
% latex plos.template
% bibtex plos.template
% latex plos.template
% latex plos.template
% dvipdf plos.template

\documentclass[10pt,draft]{article}

% amsmath package, useful for mathematical formulas
\usepackage{amsmath}
% amssymb package, useful for mathematical symbols
\usepackage{amssymb}

% graphicx package, useful for including eps and pdf graphics
% include graphics with the command \includegraphics
\usepackage{graphicx}

% cite package, to clean up citations in the main text. Do not remove.
\usepackage{cite}

\usepackage{color} 

% Use doublespacing - comment out for single spacing
%\usepackage{setspace} 
%\doublespacing


% Text layout
\topmargin 0.0cm
\oddsidemargin 0.5cm
\evensidemargin 0.5cm
\textwidth 16cm 
\textheight 21cm

% Bold the 'Figure #' in the caption and separate it with a period
% Captions will be left justified
\usepackage[labelfont=bf,labelsep=period,justification=raggedright]{caption}

% Use the PLoS provided bibtex style
\bibliographystyle{plos2009}

% Remove brackets from numbering in List of References
\makeatletter
\renewcommand{\@biblabel}[1]{\quad#1.}
\makeatother


% Leave date blank
\date{}

\pagestyle{myheadings}
%% ** EDIT HERE **


%% ** EDIT HERE **
%% PLEASE INCLUDE ALL MACROS BELOW

%% END MACROS SECTION

\begin{document}

% Title must be 150 characters or less
\begin{flushleft}
{\Large
\textbf{Digital Normalization of Short Shotgun Sequences Facilitates
{\em de novo} Sequence Assembly}
}
% Insert Author names, affiliations and corresponding author email.
\\
C. Titus Brown$^{1,\ast}$, 
Adina Howe$^{2}$,
Qingpeng Zhang$^{3}$,
Alexis B. Pyrkosz$^{4}$,
Timothy H. Brom$^{3}$
\\
\bf{1} Departments of Computer Science and Engineering/Microbiology and Molecular Genetics, Michigan State University, East Lansing, MI, USA
\\
\bf{2} Departments of Microbiology and Molecular Genetics/Crop and Soil Sciences, Michigan State University, East Lansing, MI, USA
\\
\bf{3} Department of Computer Science and Engineering, Michigan State University, East Lansing, MI, USA
\\
{\bf{4} USDA Avian Disease and Oncology Laboratory, East Lansing, MI, USA}
\\
$\ast$ E-mail: Corresponding ctb@msu.edu
\end{flushleft}

% Please keep the abstract between 250 and 300 words
\section*{Abstract}

{\bf Background:} Deep shotgun sequencing and analysis of genomes,
transcriptomes, amplified single-cell genomes, and metagenomes enable
the sensitive investigation of a wide range of biological
phenomena. However, it is difficult to deal with the volume of data
emerging from deep short-read sequencers, in part because of random
and systematic sampling variation as well as many sequencing errors
\cite{pubmed19482960,pubmed20211242}.
These challenges have led to the development of entire new classes of
short-read mapping tools such as Bowtie and BWA, as well as new de
novo assemblers such as ABySS, Velvet, SOAPdenovo, ALL-PATHS, and SGA
\cite{pubmed22068540,pubmed20211242,pubmed22156294}.
Even newer assembly strategies for dealing with transcriptomes,
single-cell genomes, and metagenomes have also emerged\cite{pubmed21572440,pubmed21926975,pubmed21685107}.  Despite these
advances, algorithms and compute capacity continue to be challenged by
the continuing improvements in sequencing technology \cite{pubmed22147368,
pubmed20835789}.
\\
\\
{\bf Methodology and Principal Findings:} We describe an approach we call
digital normalization, a single-pass computational algorithm that
reduces sampling variation and eliminates the majority of sequencing
errors from deep sequencing data. Digital normalization substantially
reduces the size of the data set and decreases the memory and time
requirements for {\em de novo} sequence assembly without significantly
impacting content of the generated contigs.  Moreover, for at least
one single-cell data set, the post-normalization assembly exhibits a significant improvement in contiguity over previous best methods.
\\
\\
{\bf Conclusions and Significance:} The digital normalization approach
yields a systematic reduction in data size and errors, and can be used
effectively prior to {\em de novo} sequence assembly.  We demonstrate its
applicability to the assembly of microbial genomes, single-cell
genomic data, and transcriptomic data.

%\bibliography{foo}

\begin{thebibliography}{10}

\bibitem{pubmed19482960}
M.~Pop.
\newblock Genome assembly reborn: recent computational challenges.
\newblock {\em Brief Bioinform}, 10(4):354--66, 2009.

\bibitem{pubmed20211242}
J.~Miller, S.~Koren, and G.~Sutton.
\newblock Assembly algorithms for next-generation sequencing data.
\newblock {\em Genomics}, 95(6):315--27, 2010.

\bibitem{pubmed22068540}
P.~Compeau, P.~Pevzner, and G.~Tesler.
\newblock How to apply de {B}ruijn graphs to genome assembly.
\newblock {\em Nat Biotechnol}, 29(11):987--91, 2011.

\bibitem{pubmed22156294}
Simpson J, Durbin R (2012) Efficient de novo assembly of large genomes using
  compressed data structures.
\newblock Genome Res.

\bibitem{pubmed21572440}
Grabherr M, Haas B, Yassour M, Levin J, Thompson D, et~al. (2011) Full-length
  transcriptome assembly from rna-seq data without a reference genome.
\newblock Nat Biotechnol 29: 644-52.

\bibitem{pubmed21926975}
Chitsaz H, Yee-Greenbaum J, Tesler G, Lombardo M, Dupont C, et~al. (2011)
  Efficient de novo assembly of single-cell bacterial genomes from short-read
  data sets.
\newblock Nat Biotechnol 29: 915-21.

\bibitem{pubmed21685107}
Y.~Peng, H.~Leung, S.~Yiu, and F.~Chin.
\newblock Meta-IDBA: a de Novo assembler for metagenomic data.
\newblock {\em Bioinformatics}, 27(13):i94--i101, 2011.

\bibitem{pubmed22147368}
S.~Salzberg, A.~Phillippy, A.~Zimin, D.~Puiu, T.~Magoc, S.~Koren, T.~Treangen,
  M.~Schatz, A.~Delcher, M.~Roberts, G.~Marcais, M.~Pop, and J.~Yorke.
\newblock Gage: A critical evaluation of genome assemblies and assembly
  algorithms.
\newblock {\em Genome Res}, 2011.

\bibitem{pubmed20835789}
Nagarajan N, Pop M (2010) Sequencing and genome assembly using next-generation
  technologies.
\newblock Methods Mol Biol 673: 1-17.


\end{thebibliography}

\end{document}

